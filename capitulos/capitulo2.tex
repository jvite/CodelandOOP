\chapter{Estado del arte}

\begin{longtable}[c]{|l|l|l|l|l|l|}
\caption{Aplicaciones existentes para aprender programación orientada a objetos}
\label{my-label}\\
\hline
\rowcolor[HTML]{34CDF9} 
{\color[HTML]{FFFFFF} Nombre} & {\color[HTML]{FFFFFF} Temas} & {\color[HTML]{FFFFFF} Juegos} & {\color[HTML]{FFFFFF} Tamaño} & {\color[HTML]{FFFFFF} Idioma} & {\color[HTML]{FFFFFF} Test} \\ \hline
\endfirsthead
%
\multicolumn{6}{c}%
{{\bfseries Table \thetable\ continued from previous page}} \\
\endhead
%
\begin{tabular}[c]{@{}l@{}}Programación\\ Orientada a Objetos\end{tabular} & \begin{tabular}[c]{@{}l@{}}-Tipo de dato anónimo\\   -Tipo abstracto\\   -SOLID\\   -RAII\\   -Proxy\\   -Programación basada en\\  prototipos\\   -Problema del diamante\\   -Principio de sustitución \\ de Liskov\\   -Principio de segregación de\\  la inerfaz\\   -Principio de responsabilidad\\  única-Objeto\\   -Método\\   -Metaclase\\   -Encapsulamiento\\   -Destructor\\   -Delegación\\   -Clase\\   -Campo\end{tabular} & No & 3.87M & Español & No \\ \hline
\begin{tabular}[c]{@{}l@{}}Object\\ Oriented Programming\end{tabular} & \begin{tabular}[c]{@{}l@{}}-Object\\   -Class\\   -Constructor\\   -Destructor\\   -Get\\   -Set\\   -toString\\   -Private Method\\   -Protected Method\\   -Public Method\\   -Inheritance\\   -Interfaces\\   -Adbstrct\\   -Plymorphism\\   -Encapsulation\end{tabular} & No & 9.64M & Ingles & No \\ \hline
\begin{tabular}[c]{@{}l@{}}Object\\ Oriented Programming\end{tabular} & \begin{tabular}[c]{@{}l@{}}-OOP in PHP - What is Object\\   -OOP in PHP - Intro to Class\\   -OOP in PHP - Inheritance\\   -OOP in PHP - Interfaces\\   -OOP in PHP - Abstract\\   -OOP in PHP - Constructor\\   -OOP in PHP - Polymorphism\\   -OOP in PHP - Encapsulation\\   -OOP in PHP - Destructor\\   -OOP in PHP - Private Method\\   -OOP in PHP - Protected Method\\   -OOP in PHP - Public Method\end{tabular} & No & 9.71M & Ingles & No \\ \hline
\begin{tabular}[c]{@{}l@{}}Object\\ Oriented Programming\end{tabular} & \begin{tabular}[c]{@{}l@{}}-Introduction to Object.\\   -Class in OOP and its detail\\  with syntax code.\\   -Abstraction concepts in OOP\\   -What is Polymorphism in OOP\\   -Inheritance concepts \\ with syntax code.\\   -Method overloading\\  vs. overriding\\   -Encapsulation concept\\   -Keywords in java\\   -Constructor in Java etc…\\   -Java Programming Statements\\  with code syntax.\end{tabular} & No & 2.8M & Ingles & No \\ \hline
OPP for Beginners & \begin{tabular}[c]{@{}l@{}}-Naming convention\\   -Object and class\\   -Method overloading\\   -Constructor\\   -Static keyword\\   -Inheritance\\   -Aggregation\\   -Method overrridong\\   -covariant return type\\   -Abstract class\\   -Package \\ -object class\end{tabular} & No & 8.1M & Ingles & Si \\ \hline
\end{longtable}

Actualmente se encuentran 5 aplicaciones similares para aprender POO, en dichas aplicaciones encontramos problemas como el lenguaje(Contenido en idioma ingles),ademas enfocan la Programación Orientada a Objetos a un lenguaje en especifíco, no hacen uso de algun método de evaluación al usuario (test) asi como la falta de ejemplos o ejercicios para reforzar el conocimiento del usuario.
En nuestro proyecto atacamos estos puntos, proporcionando una aplicación en español que posee tests, ejercicios y ejemplos enfocados a 3 distintos lenguajes (C sharp , JAVA y C++)y tomamos los puntos fuertes de cada aplicación como los temas mas frecuentes, esto nos convierte en una aplicación fuerte y competitiva.