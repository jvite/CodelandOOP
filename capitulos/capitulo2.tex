\chapter{Problematica}

\section{Ingeniería es el área con mayor tasa de abandono escolar universitario, particularmente aquellas vinculadas con la informática}
El mercado laboral sufre el dilema vocacional porque las carreras más demandadas son las menos elegidas por los estudiantes después del primero año, ya que el abandono es mayor en el área de ingeniería y especialmente en informática. El foro profesional de los ingenieros técnicos en España advierte que el déficit de profesionales en el área de software y telecomunicaciones derivará en la importancia masiva de ingenieros dentro de una década.

\section{Problemas del aprendizaje de la programación orientada a objetos}
El problema del aprendizaje de la programación orientada a objetos se manifiesta toda vez que es una materia compleja que implica la integración de muchos elementos como son el paradigma orientado a objetos, el lenguaje de programación, el entorno de desarrollo, la metodología de desarrollo, el lenguaje de modelado, los patrones de desarrollo y la lógica de programación. Por lo tanto los alumnos se encuentran ante una cantidad abrumadora de conceptos en un periodo corto de tiempo, lo que dificulta su asimilaci'on y el desarrollo de las habilidades para generar l'ineas de c'odigo como lo explica \cite{spigariol2013ensenando}:

\begin{minipage}{0.9\linewidth}
	 \vspace{5pt}
	 \begin{small}
	 	``Los docentes ve'ian en los estudiantes que el uso del lenguaje representaba una curva de aprendizaje abrupta en los primeros momentos de la materia ya que requieren el manejo de una cantidad 'amplia de conceptos antes de poder realizar algo relativamente sencillo (...). La disociaci'on entre teor'ia y pr'actica que se generaba era ciertamente contraproducente y dificultaba el proceso de aprendizaje."
	 \end{small}
\end{minipage}

Debido a que los conceptos que se tratan en la asignatura de programación orientada a objetos son muchos y en algunos casos difíciles de comprender por el alto nivel de abstracción, esto se configura como un factor que dificulta el aprendizaje, esto se ha visto reflejado por ejemplo cuando los estudiantes generalmente no distinguen entre lo que es una clase y un objeto.

Además la forma de enseñar la asignatura de programación orientada a objetos es muy similar a la de la programación estructurada, primero se tratan temas básicos del lenguaje de programación, como son la declaraciones de los tipos de datos, las estructuras de control y las sentencias de condición, para posteriormente enseñar lo que son las clases, objetos y los principales temas propios del paradigma orientada a objetos, esto contribuye a que los estudiantes sientan que continúan con el mismo paradigma de programación estructurada.

En virtud de que los estudiantes no logran asimilar este cambio de paradigmas, usan el lenguaje de programación orientada a objetos como un lenguaje de programación estructurada, por consiguiente no resuelven problemas diseñando clases.

Así pues los estudiantes al llegar a la asignatura de programación orientada a objetos se enfrentan de entrada a dos problemas, la percepción que tienen de dificultad y por otro lado el proceso de transición del paradigma estructurado hacia el paradigma orientado a objetos.

\section{Objetivo general}
Desarrollar una aplicación móvil con la herramienta de Android Studio, para que las personas interesadas en la programación puedan aprender de una mejor manera el paradigma de programación orientada a objetos, y así puedan adquirir y/o reforzar su conocimiento de este paradigma. 

\section{Objetivos espec'ificos} 
\begin{itemize}
\item Proporcionar a los estudiantes y personas interesadas en la programación, información acerca del paradigma de programación orientada a objetos.
\item Evaluar el conocimiento de los usuarios de la aplicación con exámenes al finalizar un tema.
\item Diseñar una interfaz de tal forma que el usuario pueda aprender a utilizar la aplicación en el menor tiempo posible.
\item Programar la aplicación de tal forma de que no consuma muchos recursos al usarla en un celular.
\end{itemize}